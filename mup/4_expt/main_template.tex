\documentclass{article}
\usepackage{graphicx}
\usepackage{draftwatermark}
\usepackage{hyperref}
\usepackage[a4paper, top=0.5in, bottom=0.7in, left=0.6in, right=0.6in, headheight=0.1in, footskip=0.5in]{geometry}
\usepackage{minted}

\title{MICROPROCESSOR LAB EXPERIMENT 4}
\author{GROUP - 18 \\ Deenabandhan N ee23b021 \\ Sai Harshith Gajendra ee23b069 \\ Krutarth Patel ee23b137}
\date{04 September 2024}

\SetWatermarkText{Microprocessor Lab Experiment 4}
\SetWatermarkScale{2} 
\SetWatermarkColor[gray]{0.99}

\begin{document}

\maketitle

\section*{Introduction :}
\begin{itemize}
    \item In this experiment , we are going to learn how to program the micro controller \textbf{ATmega8}.
    \item This experiment involves ,
    \begin{itemize}
        \item Introduction to assembly language.
        \item Write a program in assembly language to display the maximum and minimum of 10 numbers stored in \textbf{FLASH} memory.
        \item Write a program in assembly language to add 10 numbers stored in flash memory and store it in the register.
        \item Sort 5 numbers stored in flash memory in arbitrary order and write the final results to data memory
    \end{itemize}
    \item In this report , we have included the code of the tasks and our experience with the assembly language. 
\end{itemize}

\section*{ATmega-8 and Microchip studio :}
\begin{itemize}
    \item Atmega-8 is an 8-bit RISC single-chip microcontroller developed by Atmel.
    \item The number 8 in its name represents that it can operate 8 bits at a time while processing the information i.e in a way it represents the capacity of the microcontroller.
    \item Some features of AVR microcontroller are
    \begin{itemize}
        \item I/O ports.
        \item Internal instructions flash memory
        \item SRAM upto 16KB
        \item Timers
    \end{itemize}
    \item Flash memory is used to store the programs whatever we have written in the microchip studio.
    \item Each instruction will occupy the size of 2 bytes/16 bits in flash memory except for the instructions like \textbf{STS} , \textbf{JMP} which will occupy 4 bytes in the memory.
    \item For example the following code ,
    {\renewcommand\fcolorbox[4][]{\textcolor{cyan}{\strut#4}}
        \begin{minted}{gas}
            LDI R16,0x01
        \end{minted}
    }
    will occupy 2 bytes in the memory.
    \item Flash memory also has 32 registers (from R0 to R31) with three pointers ,
    \begin{itemize}
        \item Z pointer : R30 and R31
        \item Y pointer : R28 and R27
        \item X pointer : R25 and R26
    \end{itemize}
    \item These registers are used to hold memory in addition to having SRAM whose address starts from \textcolor{blue}{0x60}.
    \item We will see the instructions to implement the logic in the following sections.
\end{itemize}

\begin{center}
    \fbox{\Large{\textbf{\textcolor{red}{Interrupts in C}}}}
\end{center}
\section*{Introduction :}
\begin{itemize}
    \item This task involves writing a C program to transfer control 
			from a white LED(turned on) to a blinking LED on a button press
	\item pressing a button will send an interrupt signal to the program
			which will then run the subroutine we have written to turn off 
			the white LED and blink the other LED at a constant frequency.
\end{itemize}

\section*{Code}

\begin{lstlisting}[style=CStyle]

#define F_CPU 8000000UL
#include <avr/io.h>
#include<util/delay.h>
#include <avr/interrupt.h>
int main(void)
{
	DDRB=0x03; // LED connected as output
	DDRD=0x00; // input
	GICR=0x40; // setting INT0 interrupt
	SREG=0x80; // global interrupt enable
	while(1)
	{
		PORTB=0x01; // turning on LED
	}
}
ISR(INT0_vect)
{
	cli(); // disabling interrupts 
	PORTB=0x02; // switching LED
	_delay_ms(100); //blinking logic
	PORTB=0x00;
	_delay_ms(100);
	sei(); //enabling interrupts
}

\end{lstlisting}


\end{document}
