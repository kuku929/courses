\documentclass{article}
\usepackage{graphicx}
\usepackage{draftwatermark}
\usepackage{parskip}
\usepackage[a4paper, top=0.5in, bottom=0.7in, left=0.6in, right=0.6in, headheight=0.1in, footskip=0.5in]{geometry}
\usepackage{minted}

\title{MICROPROCESSOR LAB EXPERIMENT 2}
\author{Deenabandhan N ee23b021 \\ Sai Harshith Gajendra ee23b069 \\ Krutarth Patel ee23b137}
\date{18 August 2024}

\SetWatermarkText{Microprocessor Lab Experiment}
\SetWatermarkScale{2} 
\SetWatermarkColor[gray]{0.9}

\begin{document}
\maketitle
\section*{Introduction }

FPGA board : Edge Artix 7
This experiment involves 
\begin{itemize}
    \item Simulating a half-adder using Xilinx Vivado and implementing on the FPGA board. 
    \item Extending the half-adder design to a full-adder, simulating it and implementing on the FPGA board. 
    \item Designing a 4-bit ripple-carry adder and implementing on the
            FPGA board.
\end{itemize}
\section*{Xilinx Vivado}
\begin{itemize}
    \item We were introduced to Xilinx vivado , a software that is used to synthesize and analyse the hardware description designs.
    \item The procedures that we followed are ,
    \begin{itemize}
        \item Create a project with source file as our Verilog code.
        \item Add constraints to the ports that we have defined in the Verilog code.
        \item Run the synthesis to check whether our Verilog code has any error in it.
        \item We can open the Schematics to see the design that we have coded in Verilog.
        \item After running synthesis , we can run simulation using our testbench and check for logical errors.
        \item Once we confirm there is no logical/syntax error , we can run the implementation which basically implements our hardware description in the board which we have chosen while creating the project.
        \item After the implementation is done , we can generate bitstream which is basically a file that contains the configuration information for an FPGA.
        \item Once we successfully generate the bitstream , we can connect the target source and program it with the generated bitstream.
    \end{itemize}
    \item In this report , we have included 
    \begin{itemize}
        \item Verilog code for module instantiation.
        \item We have included both data flow as well as gate level modelling here.
        \item The Schematics generated from Xilinx Vivado.
        \item The Constraints file generated for FPGA.
        \item The testbench and simulation generated
        \item The analysis of reports obtained from Xilinx Vivado.
    \end{itemize}
\end{itemize}
\section*{Johnson Counter}

% Data-------------------------------------------------
\subsection*{Data flow model}
\begin{minted}{verilog}
module decoder(input[2:0] cntr,output [7:0] led);
	reg [6:0] val;
	assign led={1'b1,~(val)};
	
	always@(cntr)
	begin
		case(cntr)
		3'd0 : val=7'b0111111;
		3'd1 : val=7'b0000110;
		3'd2 : val=7'b1011011;
		3'd3 : val=7'b1001111;
		3'd4 : val=7'b1100110;
		3'd5 : val=7'b1101101;
		3'd6 : val=7'b1111101;
		3'd7 : val=7'b0000111;
		endcase
	end
endmodule

module Johnson_count(input clk_in,reset,input [3:0] digit,output [7:0] led);
	wire [2:0] cntr;
	wire q0,q1,q2;
	wire q0_bar,q1_bar,q2_bar;
	wire clk_out;
	
	assign digit=4'b0001;
	
	clk_gen c1(clk_in,reset,clk_out);
	
	D_FF d1(q0_bar,clk_out,reset,q2,q2_bar);
	D_FF d2(q2,clk_out,reset,q1,q1_bar);
	D_FF d3(q1,clk_out,reset,q0,q0_bar);
	
	assign cntr={q2,q1,q0};
	
	decoder de1(cntr,led);

endmodule
\end{minted}
\subsubsection*{Schematics : }
\begin{figure}[H]
    \centering
    \includegraphics[width=150 mm]{
		johnson.png
	}
    \caption{Schematics of the Data flow modelling}
\end{figure}

% Constraints-------------------------------------------------
\subsection*{Constraints on ports of FPGA }
\begin{minted}{csharp}
set_property IOSTANDARD LVCMOS33 [get_ports {digit[3]}]
set_property IOSTANDARD LVCMOS33 [get_ports {digit[2]}]
set_property IOSTANDARD LVCMOS33 [get_ports {digit[1]}]
set_property IOSTANDARD LVCMOS33 [get_ports {digit[0]}]
set_property IOSTANDARD LVCMOS33 [get_ports {led[7]}]
set_property IOSTANDARD LVCMOS33 [get_ports {led[6]}]
set_property IOSTANDARD LVCMOS33 [get_ports {led[5]}]
set_property IOSTANDARD LVCMOS33 [get_ports {led[4]}]
set_property IOSTANDARD LVCMOS33 [get_ports {led[3]}]
set_property IOSTANDARD LVCMOS33 [get_ports {led[2]}]
set_property IOSTANDARD LVCMOS33 [get_ports {led[1]}]
set_property IOSTANDARD LVCMOS33 [get_ports {led[0]}]
set_property IOSTANDARD LVCMOS33 [get_ports clk_in]
set_property IOSTANDARD LVCMOS33 [get_ports reset]
set_property PACKAGE_PIN F2 [get_ports {digit[0]}]
set_property PACKAGE_PIN E1 [get_ports {digit[1]}]
set_property PACKAGE_PIN G5 [get_ports {digit[2]}]
set_property PACKAGE_PIN G4 [get_ports {digit[3]}]
set_property PACKAGE_PIN H1 [get_ports {led[7]}]
set_property PACKAGE_PIN H2 [get_ports {led[6]}]
set_property PACKAGE_PIN J4 [get_ports {led[5]}]
set_property PACKAGE_PIN J5 [get_ports {led[4]}]
set_property PACKAGE_PIN H4 [get_ports {led[3]}]
set_property PACKAGE_PIN H5 [get_ports {led[2]}]
set_property PACKAGE_PIN G1 [get_ports {led[1]}]
set_property PACKAGE_PIN G2 [get_ports {led[0]}]
set_property PACKAGE_PIN N11 [get_ports clk_in]
set_property PACKAGE_PIN L5 [get_ports reset]
\end{minted}

% Simulation-------------------------------------------------
\subsection*{Simulation :}
\begin{figure}[H]
    \centering
    \includegraphics[width=180 mm]{
		gtkwav-johnson.png
	}
\end{figure}

% Report-------------------------------------------------
\subsection*{Reports :}
\subsubsection*{Resources utilized : }
\begin{itemize}
    \item 47 \textbf{SLICE LUT} were used, where LUT is used as a logic and not memory.
	\item 37 Slice Registers were used as Flip Flop.
\end{itemize}
\subsubsection*{Clock :}
\begin{itemize}
	\item only 1 Global Clock was utilized. 
\end{itemize}
\begin{center}
\begin{tabular}{|c|c|c|c|}\hline
\textit{Clock Region Name} & \textit{Global Clock (Used)} & \textbf{FF (Used)} & \textbf{LUTM (Used)}\\
\hline
\textit{X0Y0} & \textit{0} & \textit{0} & \textit{0}\\
\textit{X1Y0} & \textit{0} & \textit{0} & \textit{0}\\
\textit{X0Y1} & \textit{0} & \textit{0} & \textit{0}\\
\textit{X1Y1} & \textit{1} & \textit{37} & \textit{6}\\
\textit{X0Y2} & \textit{0} & \textit{0} & \textit{0}\\
\textit{X1Y2} & \textit{0} & \textit{0} & \textit{0}\\
\hline
\end{tabular}
\end{center}
\subsubsection*{Time delays}
\begin{center}
\begin{tabular}{|c|c|c|}\hline
\textit{Type of Path} & \textit{Path taken} & \textbf{Time delay (ns)}\\
\hline
\textit{Max Delay path} 
    & \textbf{} \(\rightarrow\) \textbf{} & \\\cline{2-3}
    & \textbf{} \(\rightarrow\) \textbf{}  & \\\cline{2-3}
    & \textbf{} \(\rightarrow\) \textbf{}  & \\\cline{2-3}
    & \textbf{} \(\rightarrow\) \textbf{}  & \\\cline{2-3}
    & \textbf{} \(\rightarrow\) \textbf{}  & \\\cline{2-3}
    \hline
\textit{Min Delay path}
    & \textbf{} \(\rightarrow\) \textbf{} & \\\cline{2-3}
    & \textbf{} \(\rightarrow\) \textbf{}  & \\\cline{2-3}
    & \textbf{} \(\rightarrow\) \textbf{}  & \\\cline{2-3}
    & \textbf{} \(\rightarrow\) \textbf{}  & \\\cline{2-3}
    & \textbf{} \(\rightarrow\) \textbf{}  & \\\cline{2-3}
    \hline
\end{tabular}
\end{center}
\subsubsection*{Power consumed}
\begin{itemize}
    \item The power consumed by FPGA is \textbf{0.325}
\end{itemize}

\end{document}

