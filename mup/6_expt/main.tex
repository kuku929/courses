\documentclass{article}
\usepackage{graphicx}
\usepackage{draftwatermark}
\usepackage{hyperref}
\usepackage[a4paper, top=0.5in, bottom=0.7in, left=0.6in, right=0.6in, headheight=0.1in, footskip=0.5in]{geometry}
\usepackage{minted}
\usepackage{listings}

\title{MICROPROCESSOR LAB EXPERIMENT 5}
\author{GROUP - 18 \\ Deenabandhan N ee23b021 \\ Sai Harshith Gajendra ee23b069 \\ Krutarth Patel ee23b137}
\date{27 September 2024}

\SetWatermarkText{Microprocessor Lab Experiment 5}
\SetWatermarkScale{2} 
\SetWatermarkColor[gray]{0.99}

\definecolor{mGreen}{rgb}{0,0.6,0}
\definecolor{mGray}{rgb}{0.5,0.5,0.5}
\definecolor{mPurple}{rgb}{0.58,0,0.82}
\definecolor{backgroundColour}{rgb}{0.95,0.95,0.92}

\definecolor{codegreen}{rgb}{0,0.6,0}
\definecolor{codegray}{rgb}{0.5,0.5,0.5}
\definecolor{codepurple}{rgb}{0.58,0,0.82}
\definecolor{backcolour}{rgb}{0.95,0.95,0.92}}

\lstdefinestyle{CStyle}{
    backgroundcolor=\color{backcolour},   
    commentstyle=\color{codegreen},
    keywordstyle=\color{magenta},
    numberstyle=\tiny\color{codegray},
    stringstyle=\color{codepurple},
    basicstyle=\ttfamily\scriptsize,
    breakatwhitespace=false,         
    breaklines=true,                 
    captionpos=b,                    
    keepspaces=true,                 
    numbers=left,                    
    numbersep=5pt,                  
    showspaces=false,                
    showstringspaces=false,
    showtabs=false,                  
    tabsize=2,
    language=c
}
\lstset{style=CStyle}

\begin{document}
\maketitle
\section*{Introduction :}
\begin{itemize}
	\item In this experiment , we are going to learn about interrupts and subroutines in C and Assembly programs using the \textbf{ATmega8}.
    \item This experiment involves ,
    \begin{itemize}
        \item Introduction to Interrupt handling.
		\item Writing a program in C to transfer control from a white LED(turned on) to a blinking LED on a button press
        \item Writing a program in Assembly to blink an LED upon receiving an interrupt in the form of a button press
        \item Changing the code in the above task to transfer control from a white LED(turned on) to a blinking LED on a button press
    \end{itemize}
    \item In this report , we have included the code of the tasks and our experience with C and Assembly. 
\end{itemize}

\section*{ATmega-8 and Microchip studio :}
\begin{itemize}
    \item Atmega-8 is an 8-bit RISC single-chip microcontroller developed by Atmel.
    \item The number 8 in its name represents that it can operate 8 bits at a time while processing the information i.e in a way it represents the capacity of the microcontroller.
    \item Some features of AVR microcontroller are
    \begin{itemize}
        \item I/O ports.
        \item Internal instructions flash memory
        \item SRAM upto 16KB
        \item Timers
    \end{itemize}
    \item The AtMega8 microcontroller has a total of 32 8-bit registers and 23 I/O pins.
    \subsection*{Atmega8 microcontroller pin diagram :}
    \begin{itemize}
        \item The pin diagram of Atmega8 microcontroller is ,
        \begin{figure}[H]
            \centering
            \includegraphics[width=0.5\linewidth]{pin_atm.png}
        \end{figure}
        \item It has 3 ports PortB,PortC and PortD.
        \item Each port acts as a bidirectional buffer that could carry both input and output values with specific address.
        \item The registers that are associated with these ports are 
        \begin{itemize}
            \item \textbf{DDRX} - Register to mention whether the particular pin is input/output.
            Eg : DDRD=0x0F means , first 8 pins are output pins and the rest are input pins.
            \item \textbf{PORTX} - Register to mention the output to be given through the pin.
            Eg : PORTC=0xF0 means that the first 8 pins of Port C are set to logic low and the rest of them are set to logic high.
            \item \textbf{PINX} - Register that is used to store the value thta is given as input in the pins.
            Eg : a=PINB means that whatever input that is given at port B is given to the variable a.
        \end{itemize}
        \item In addition to these ports it also supports interrupt operations which is an important instruction in any microcontroller.
    \end{itemize}
    \subsection*{Libraries used in the C code}
    \begin{itemize}
        \begin{lstlisting}[style=CStyle]
#include <avr/io.h>\end{lstlisting}
        \item The above library is used to include standard avr commands like \textbf{DDRD , PORTC , PINB}

            \begin{lstlisting}[style=CStyle]
#include <util/delay.h>\end{lstlisting}
        \item The above library is used to include time delays using the function , \begin{lstlisting}[style=CStyle]
delay_ms(100) //includes 100ms delay \end{lstlisting}
    \end{itemize}
\end{itemize}

\begin{center}
    \fbox{\Large{\textbf{\textcolor{red}{Bonus Question}}}}
\end{center}
\section*{Introduction :}
\begin{itemize}
    \item This task involves taking two 4-bit numbers from PORT-D and displaying their
			sum using LEDs connected to PORT-C.
\end{itemize}

\section*{Code}

\begin{lstlisting}[style=CStyle]
#include <avr/io.h>
#include <stdint.h>
#include <util/delay.h>
int main(void)
{
	DDRC = 0x1F; //output port
	DDRD = 0x00; //input port
	
	while(1)
	{
		unsigned char data = PIND; //receiving data 

		// we get the 4-bit numbers as one 8-bit number
		// eg: number1 ->0101 1110<- number2
		PORTC = (data>>4)+ (data&15); 
		_delay_ms(10);
	}
}

\end{lstlisting}
\section*{Process}
\begin{itemize}
    \item Getting two 4-bit numbers from PORT-C.
    \item Adding the two numbers.
    \item Displaying the result using LEDs.
\end{itemize}

\end{document}
