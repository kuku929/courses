\documentclass{article}
\usepackage{graphicx}
\usepackage{draftwatermark}
\usepackage{parskip}
\usepackage[a4paper, top=0.5in, bottom=0.7in, left=0.6in, right=0.6in, headheight=0.1in, footskip=0.5in]{geometry}
\usepackage{minted}

\title{MICROPROCESSOR LAB EXPERIMENT 2}
\author{Deenabandhan N ee23b021 \\ Sai Harshith Gajendra ee23b069 \\ Krutarth Patel ee23b137}
\date{18 August 2024}

\SetWatermarkText{Microprocessor Lab Experiment}
\SetWatermarkScale{2} 
\SetWatermarkColor[gray]{0.9}

\begin{document}
\maketitle
\section*{Introduction }

FPGA board : Edge Artix 7
This experiment involves 
\begin{itemize}
    \item Simulating a half-adder using Xilinx Vivado and implementing on the FPGA board. 
    \item Extending the half-adder design to a full-adder, simulating it and implementing on the FPGA board. 
    \item Designing a 4-bit ripple-carry adder and implementing on the
            FPGA board.
\end{itemize}
\section*{Xilinx Vivado}
\begin{itemize}
    \item We were introduced to Xilinx vivado , a software that is used to synthesize and analyse the hardware description designs.
    \item The procedures that we followed are ,
    \begin{itemize}
        \item Create a project with source file as our Verilog code.
        \item Add constraints to the ports that we have defined in the Verilog code.
        \item Run the synthesis to check whether our Verilog code has any error in it.
        \item We can open the Schematics to see the design that we have coded in Verilog.
        \item After running synthesis , we can run simulation using our testbench and check for logical errors.
        \item Once we confirm there is no logical/syntax error , we can run the implementation which basically implements our hardware description in the board which we have chosen while creating the project.
        \item After the implementation is done , we can generate bitstream which is basically a file that contains the configuration information for an FPGA.
        \item Once we successfully generate the bitstream , we can connect the target source and program it with the generated bitstream.
    \end{itemize}
    \item In this report , we have included 
    \begin{itemize}
        \item Verilog code for module instantiation.
        \item We have included both data flow as well as gate level modelling here.
        \item The Schematics generated from Xilinx Vivado.
        \item The Constraints file generated for FPGA.
        \item The testbench and simulation generated
        \item The analysis of reports obtained from Xilinx Vivado.
    \end{itemize}
\end{itemize}
\section*{Wallace Multiplier}

% Data-------------------------------------------------
\subsection*{Data flow model}
\begin{minted}{verilog}
module unsigned_mult(output [7:0] m , input [3:0] a,b);
	wire [3:0]p[0:3];
	genvar i,j;
	generate
	for(i=0;i<4;i=i+1)
	begin
		for(j=0;j<4;j=j+1)
		begin
			assign p[i][j]=a[j]*b[i];
		end
	end
	endgenerate
	wire s0,s1,s2,s3,s4;
	wire c0,c1,c2,c3,c4;	
	
	wire k1,l1,k2,l2;
	
	assign m[0]=p[0][0];
	
	half_adder h0(k1,l1,p[3][0],p[2][1]);
	
	half_adder h1(k2,l2,p[1][3],p[2][2]);
	
	half_adder h2(s0,c0,p[0][1],p[1][0]);
	
	FA_S f1(s1,c1,p[2][0],p[1][1],p[0][2]);
	
	FA_S f2(s2,c2,p[0][3],p[1][2],k1);
	
	FA_S f3(s3,c3,p[3][1],k2,l1);
	
	FA_S f4(s4,c4,p[2][3],p[3][2],l2);
	
	wire u1,u2,u3,u4;
	
	assign m[1]=s0;
	
	half_adder h3(m[2],u1,s1,c0);
	
	FA_S f5(m[3],u2,u1,s2,c1);
	
	FA_S f6(m[4],u3,u2,s3,c2);
	
	FA_S f7(m[5],u4,u3,s4,c3);
	
	FA_S f8(m[6],m[7],u4,p[3][3],c4);
endmodule
\end{minted}

% Schematics------------------------------------------------
\subsubsection*{Schematics : }
\begin{figure}[H]
    \centering
    \includegraphics[width=150 mm]{
		wallace.png
	}
    \caption{Schematics of the Data flow modelling}
\end{figure}

% Constraints-------------------------------------------------
\subsection*{Constraints on ports of FPGA }
\begin{minted}{csharp}
set_property IOSTANDARD LVCMOS33 [get_ports {a[3]}]
set_property IOSTANDARD LVCMOS33 [get_ports {a[2]}]
set_property IOSTANDARD LVCMOS33 [get_ports {a[1]}]
set_property IOSTANDARD LVCMOS33 [get_ports {a[0]}]
set_property PACKAGE_PIN L5 [get_ports {a[0]}]
set_property PACKAGE_PIN L4 [get_ports {a[1]}]
set_property PACKAGE_PIN M4 [get_ports {a[2]}]
set_property PACKAGE_PIN M2 [get_ports {a[3]}]
set_property PACKAGE_PIN M1 [get_ports {b[0]}]
set_property PACKAGE_PIN N3 [get_ports {b[1]}]
set_property PACKAGE_PIN N2 [get_ports {b[2]}]
set_property PACKAGE_PIN N1 [get_ports {b[3]}]
set_property PACKAGE_PIN J3 [get_ports {m[0]}]
set_property PACKAGE_PIN H3 [get_ports {m[1]}]
set_property PACKAGE_PIN J1 [get_ports {m[2]}]
set_property PACKAGE_PIN K1 [get_ports {m[3]}]
set_property PACKAGE_PIN L3 [get_ports {m[4]}]
set_property PACKAGE_PIN L2 [get_ports {m[5]}]
set_property PACKAGE_PIN K3 [get_ports {m[6]}]
set_property PACKAGE_PIN K2 [get_ports {m[7]}]
set_property IOSTANDARD LVCMOS33 [get_ports {b[3]}]
set_property IOSTANDARD LVCMOS33 [get_ports {b[2]}]
set_property IOSTANDARD LVCMOS33 [get_ports {b[1]}]
set_property IOSTANDARD LVCMOS33 [get_ports {b[0]}]
set_property IOSTANDARD LVCMOS33 [get_ports {m[7]}]
set_property IOSTANDARD LVCMOS33 [get_ports {m[6]}]
set_property IOSTANDARD LVCMOS33 [get_ports {m[5]}]
set_property IOSTANDARD LVCMOS33 [get_ports {m[4]}]
set_property IOSTANDARD LVCMOS33 [get_ports {m[3]}]
set_property IOSTANDARD LVCMOS33 [get_ports {m[2]}]
set_property IOSTANDARD LVCMOS33 [get_ports {m[1]}]
set_property IOSTANDARD LVCMOS33 [get_ports {m[0]}]}

\end{minted}

% Report-------------------------------------------------
\subsection*{Reports :}
\subsubsection*{Resources utilized : }
\begin{center}
\begin{tabular}{|c|c|c|}\hline
\textit{Ref Name} & \textit{Used} & \textit{Functional category}\\
\hline
\textit{OBUF} & \textit{8} & \textbf{IO}\\
\hline
\textit{IBUF} & \textit{8} & \textbf{IO}\\
\hline
\textit{LUT2} & \textit{2} & \textbf{LUT}\\
\hline
\textit{LUT3} & \textit{2} & \textbf{LUT}\\
\hline
\textit{LUT4} & \textit{3} & \textbf{LUT}\\
\hline
\textit{LUT5} & \textit{7} & \textbf{LUT}\\
\hline
\textit{LUT6} & \textit{6} & \textbf{LUT}\\
\hline
\end{tabular}
\end{center}
\subsubsection*{Time delays}
\begin{center}
\begin{tabular}{|c|c|c|}\hline
\textit{Type of Path} & \textit{Path taken} & \textbf{Time delay (ns)}\\
\hline
\textit{Max Delay path} 
					  & \textbf{b[6]} \(\rightarrow\) \textbf{m[6]} & \textbf{11.016}  \\\cline{2-3}
					  & \textbf{b[3]} \(\rightarrow\) \textbf{m[3]}  & \textbf{10.913} \\\cline{2-3}
					  & \textbf{b[3]} \(\rightarrow\) \textbf{m[7]}  & \textbf{10.877} \\\cline{2-3}
    \hline
\textit{Min Delay path}
					  & \textbf{a[1]} \(\rightarrow\) \textbf{m[2]} & \textbf{2.275}  \\\cline{2-3}
					  & \textbf{b[1]} \(\rightarrow\) \textbf{m[7]}  & \textbf{2.297} \\\cline{2-3}
					  & \textbf{b[1]} \(\rightarrow\) \textbf{m[6]}  & \textbf{2.332} \\\cline{2-3}
    \hline
\end{tabular}
\end{center}
\subsubsection*{Power consumed}
\begin{itemize}
    \item The power consumed by FPGA is \textbf{0.325 W}
\end{itemize}

\end{document}

