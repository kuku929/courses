\documentclass{article}
\usepackage{amsmath}
\usepackage{graphicx}
\usepackage{placeins}
\usepackage{listings}
\usepackage{xcolor}
\usepackage{hyperref}

\definecolor{codegreen}{rgb}{0,0.6,0}
\definecolor{codegray}{rgb}{0.5,0.5,0.5}
\definecolor{codepurple}{rgb}{0.58,0,0.82}
\definecolor{backcolour}{rgb}{0.95,0.95,0.92}

\lstdefinestyle{mystyle}{
    backgroundcolor=\color{backcolour},   
    commentstyle=\color{codegreen},
    keywordstyle=\color{magenta},
    numberstyle=\tiny\color{codegray},
    stringstyle=\color{codepurple},
    basicstyle=\ttfamily\scriptsize,
    breakatwhitespace=false,         
    breaklines=true,                 
    captionpos=b,                    
    keepspaces=true,                 
    numbers=left,                    
    numbersep=5pt,                  
    showspaces=false,                
    showstringspaces=false,
    showtabs=false,                  
    tabsize=2
}

\lstset{style=mystyle}


\title{Imaging Using Microphone Arrays}
\author{EE23B137 Krutarth Patel}
\date{\today}

\begin{document}

\maketitle

\section{Objective}
\textbf{NOTE}: Please go through the code file to get the answers to the questions posed in the assignment. This report is mainly for the purposes of showing the graphical output
of the code. I have explained things in much more detail there \\
\\
The objective of this project is to simulate an acoustic imaging system using a microphone array and to reconstruct an image of a point obstacle using the Delay-And-Sum (DAS) algorithm. 

\section{Usage}
the code sequentially displays various plots pertaining to the questions asked. To step through them, close the window once you are done verifying the results
shown this will cause a new window to open with the next plot. I have attached images of all the important plots in this report for your ready reference as well.
\section{System Setup and Parameters}
\subsection{Microphone Array Setup}
\begin{itemize}
    \item Number of microphones, $N_{\text{mics}}$: 64
    \item Microphones are aligned along the y-axis with a spacing (pitch) of 0.1 units.
    \item Source position: $(0, 0)$.
\end{itemize}

\subsection{Sound Wave Parameters}
\begin{itemize}
    \item Sampling rate proxy ($\text{dist\_per\_samp}$): 0.1 units.
    \item Speed of sound, $C$: 2.0 units.
    \item Obstacle position: $(3, -1)$.
    \item Sinc Pulse Narrowness, $\text{SincP}$: 10.0, which controls the pulse frequency.
\end{itemize}

\begin{figure}[!h]
	\centering
	\includegraphics[height=8cm]{single_obst.png}
    \caption{basic implmentation}
\end{figure}


\subsection{Effect of the \texttt{SincP} Parameter}
The $\text{SincP}$ parameter determines the frequency and narrowness of the sinc pulse. Higher values of $\text{SincP}$ produce narrower, higher-frequency pulses. Thinner pulses provide sharper peaks, making it easier to distinguish between closely spaced objects, which is beneficial for image reconstruction.

\section{Results and Observations}
\subsection{Example Sinc Pulses}
Increasing $\text{SincP}$ results in higher frequency pulses, creating narrower peaks. Narrower pulses allow for sharper image reconstruction, as they improve spatial resolution, making it easier to differentiate between closely spaced objects.

\subsection{Maximum Intensity Position and Scaling}
\begin{itemize}
    \item \textbf{Observed Position:} The DAS reconstruction locates the obstacle at approximately $(30, 22)$ in the new coordinate system.
    \item \textbf{Explanation:} This discrepancy arises due to scaling differences. Let $(X', Y')$ denote the DAS coordinate system and $(X, Y)$ the absolute coordinates.
    \begin{equation}
    X' = \frac{X}{\text{dist\_per\_samp}}, \quad Y' = \frac{N}{2} + \frac{Y}{\text{pitch}}
    \end{equation}
    The actual coordinates $(3, -1)$ align with approximately $(30, 22)$ in the scaled system used for reconstruction.
\end{itemize}

\subsection{Bounds on X and Y coordinates}

I have prepared a quick graph display on desmos to show this graphically, 
please refer to it \href{https://www.desmos.com/calculator/zar4naicoh}{\textbf{here}}.\\ 
The maximum distance a mic can detect is $Nsamp * dist\_per\_samp$, which is 20. Therefore we plot
ellipses with major axis length = 20. and get the maximum x-coordinate and y-coordinate from all the
Nmics ellipses( the other focus is the source ).
it is very easy to see, that the maximum y-coordinate is 11.6 and x-coordinate is 10. for the constants used 
in the report. In general,
   \begin{gather}
	   maxy = \frac{Nsamp * dist\_per\_samp}{2} + \frac{Nmics * pitch}{4}\\
	   maxx = \frac{Nsamp * dist\_per\_samp}{2}
   \end{gather}
\begin{figure}[!h]
	\centering
	\includegraphics[height=8cm]{desmos-graph.png}
\end{figure}

\FloatBarrier

\subsection{Effects of Changing Speed of Sound, $C$}
Decreasing $C$ without adjusting the sampling rate reduces $\text{dist\_per\_samp}$, allowing the system to differentiate between closer objects. This creates a sharper image, as the system can more accurately distinguish real objects from false positives.

\subsection{Impact of Varying Number of Microphones and Samples}
The following combinations of $N_{\text{mics}}$ and $N_{\text{samp}}$ were tested to understand their impact on the reconstructed image:
\begin{itemize}
    \item $N_{\text{mics}} = \{8, 32, 64\}$ and $N_{\text{samp}} = \{50, 100, 200\}$
    \item \textbf{Observations:} Increasing $N_{\text{mics}}$ provides more data points, leading to a clearer image. Higher $N_{\text{samp}}$ improves resolution along the x-axis, yielding a more detailed image.
\end{itemize}

\begin{figure}[!h]
	\centering
	\includegraphics[height=10cm]{plot_grid.png}
\end{figure}

\FloatBarrier

\section{Reconstruction from File Data}
To verify the DAS algorithm's robustness, files \texttt{rx2.txt} and \texttt{rx3.txt} were used. Each file contains pre-recorded data for all microphones, which were processed and displayed using the same DAS-based reconstruction approach.

\begin{figure}[!h]
	\centering
	\includegraphics[height=6cm]{rx2.png}
	\caption{rx2}
\end{figure}

\begin{figure}[!h]
	\centering
	\includegraphics[height=6cm]{rx3.png}
	\caption{rx3}
\end{figure}

\end{document}
