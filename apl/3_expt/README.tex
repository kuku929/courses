
\documentclass{report}
\usepackage{graphicx} % Required for inserting images
\graphicspath{ {./img/} }
\usepackage{listings}
\usepackage{xcolor}
\usepackage{placeins}
\usepackage[a4paper, top=0.1in, bottom=0.1in, left=1in, right=1in]{geometry}
\definecolor{codegreen}{rgb}{0,0.6,0}
\definecolor{codegray}{rgb}{0.5,0.5,0.5}
\definecolor{codepurple}{rgb}{0.58,0,0.82}
\definecolor{backcolour}{rgb}{0.95,0.95,0.92}

\lstdefinestyle{mystyle}{
    backgroundcolor=\color{backcolour},   
    commentstyle=\color{codegreen},
    keywordstyle=\color{magenta},
    numberstyle=\tiny\color{codegray},
    stringstyle=\color{codepurple},
    basicstyle=\ttfamily\scriptsize,
    breakatwhitespace=false,         
    breaklines=true,                 
    captionpos=b,                    
    keepspaces=true,                 
    numbers=left,                    
    numbersep=5pt,                  
    showspaces=false,                
    showstringspaces=false,
    showtabs=false,                  
    tabsize=2
}

\lstset{style=mystyle}
\title{Estimating the Black Body curve}
\author{Krutarth Patel ee23b137}
\date{14th september 2024}

\begin{document}

\maketitle

\section*{Introduction}
This report aims to estimate the various constants in the black 
body curve using raw data.

The equation for the curve is given by:
\[I(\nu, T) = \frac{2 \pi h c^2}{\lambda^5} \cdot \frac{1}{e^{\frac{h c}{\lambda k_B T}} - 1}\]
\begin{itemize}
	\item $I(\lambda ,T)$ is the intensity of radiation at wavelength $\lambda$ and temperature T,
	\item $h$ is Planck's constant,
	\item $\lambda$ is the wavelength of radiation,
	\item $c$ is the speed of light in a vacuum,
	\item $k_B$ is the Boltzmann's constant,
	\item $T$ is the absolute temperature of the black body.
\end{itemize}

We are interested in estimating the values of \[h, c, k_B, T\]

\section*{Data}
the following samples are provided to us: 
\begin{figure}[htp]
	\centering
	\includegraphics[width=12cm]{d1.png}
    \caption{Sample 1}
\end{figure}
\begin{figure}[htp]
	\centering
	\includegraphics[width=12cm]{d3.png}
    \caption{Sample 3}
\end{figure}
\begin{figure}[htp]
	\centering
	\includegraphics[width=12cm]{d2.png}
    \caption{Sample 2}
\end{figure}
\begin{figure}[htp]
	\centering
	\includegraphics[width=12cm]{d4.png}
    \caption{Sample 4}
\end{figure}

\section*{No Initial guess}
To begin with we will try to fit the sample points with no prior estimation given. We quickly run into a problem though, trying to fit without an initial guess leads to a float overflow error. This is expected since Scipy assumes an initial value of 1 for all parameters. Assuming 
\[h, c, k_B, T = 1\] 
and 
\[\lambda = 10^{-7}\] 
the approximate value of \[I(\lambda, T) = \frac{2 \pi}{10^{-35}} \cdot \frac{1}{e^{10^7} - 1}\]
which for obvious reasons cannot be stored using the precision of a 32-bit floating point number. $e^{10^7}$ is causing the overflow, thus we can \textbf{normalize} the samples. Multiplying each input value with $10^6$ avoids the overflow error and allows the algorithm to begin iterating.

\section*{Algorithm}
Now is a good time to talk about the tool I am using to fit. my first instinct was to mess with the data using \texttt{gnuplot}, it is a command-line plotting tool which is very easy to use. gnuplot has a fitting tool as well, so I tried it first. 

Passing the normalized data to the fitting tool in \texttt{gnuplot} using the following command: 

\begin{lstlisting}[language=matlab]
	fit f(x) 'd1_normalized.txt' via h,k,c,t
\end{lstlisting}	

But, this does not help much. We get the following fit:
\begin{figure}[!h]
	\centering
	\includegraphics[height=8cm]{gnuplot_first.png}
    \caption{first attempt}
\end{figure}

Now, to converge the fit properly, I used an approximation of the 
black body curve: \[I(\nu, T) = \frac{2 \pi h c^2}{\lambda^5} \cdot ({e^{\frac{-h c}{\lambda k_B T}} - 1})\]

This gives: 
\begin{figure}[!h]
	\centering
	\includegraphics[height=8cm]{gnuplot_second.png}
    \caption{using approximation}
\end{figure}
\FloatBarrier
Alternating between this approximation and the real formula, I was able to get the following curve:

\begin{figure}[!h]
	\centering
	\includegraphics[height=8cm]{gnuplot_fit2.png}
    \caption{no initial guess(note the values in x axis range from 0-5)}
\end{figure}

\FloatBarrier
\textbf{NOTE}: This can be accomplished using SciPy as well, The code I have submitted will approximate without a guess and with a guess and show the results. 
\section*{With Initial guess}
For this section I have used the SciPy libraries \texttt{curve\_fit} function.

The code for the same looks like this: 
\begin{lstlisting}[language=Python]

import scipy
import numpy as np
import matplotlib.pyplot as plt


def f(x, h, c, k, t):
    return (2 * h * c * c / x**5) / 
			(np.exp(1 * h * c / (x * k * t)) - 1)


def main():
    xdata = []
    ydata = []
    with open("data/d3.txt", "r") as fin:
        for line in fin.readlines():
            x, y = line.split(",")
            xdata.append(float(x))
            ydata.append(float(y))

    #fitting
    initial_guess = [6.62 * 1e-34, 3 * 1e3, 1.38 * 1e-23, 1]
    params, cov = scipy.optimize.curve_fit(
        f, xdata, ydata, initial_guess, nan_policy="omit"
    )

    #getting prediction	
    ypred = []
    for val in xdata:
        ypred.append(
					f(val, params[0], params[1], 
				    params[2], params[3])
							      )

    #plotting
    plt.title("Curve fitting")
    plt.xlabel("wavelength (m)")
    plt.ylabel("power density (Watts/m^3)")
    plt.plot(xdata, ypred, label="fitted curve")
    plt.plot(
        xdata, ydata, markersize=2, marker=".", linestyle="None", label="data sample"
    )
    plt.legend()
    plt.show()


if __name__ == "__main__":
    main()

\end{lstlisting}
Since there are three known constants $h, c, k_B$, we can give the initial guess of any combination of these, starting with:
\[h=6.62\times10^{-34}, c=3\times10^3, k_B=1.38\times10^{-23}\]
\begin{figure}[!h]
	\centering
	\includegraphics[]{scipy_fit.png}
\end{figure}

\FloatBarrier
And the final constants are: \[ h=2.17146884*10^{-31}, c=1.61328417*10^7, k_B = -3.53152592*10^{-21},  t=-2.78747897*10^2 \]

\section*{Using the program}
You can reproduce these results with the python file I have attached with my submission. Please keep the following in mind while using the program:
\begin{itemize}
	\item pass the filename as a positional argument
	\item As mentioned earlier, I normalize the input data by multiplying with 
			a constant. To get a correct fit, One will have to play around with this normalization constant. For the purpose of this assignment please use the following values:
		\begin{itemize}
			\item for \texttt{d1.txt} use $10^6$
			\item for \texttt{d3.txt} use $10^5$
		\end{itemize}
Thus, running the program would look something like this: 
\begin{lstlisting}[language=bash]
	$ python3 ee23b137.py d1.txt -normalize 1e6
\end{lstlisting}
\end{itemize}
\end{document}

