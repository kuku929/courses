\input{preamble}
\input{format}
\input{commands}
\usepackage{amssymb}
\usepackage{algpseudocode}

\begin{document}

\begin{Large}
    \textsf{\textbf{CS6851: Distributed Algorithms}}
    
    Assignment 1
\end{Large}

\vspace{1ex}

\textsf{\textbf{Student:}} \text{Krutarth Patel}, \href{ee23b137@smail.iitm.ac.in}{ee23b137@smail.iitm.ac.in}\\

\begin{problem}{}{problem-label}
\begin{enumerate}[(a)]
    \item Let’s revisit the lower bound for 2-coloring paths. We showed a lower bound for a graph with
$4t+ 2$ vertices, by combining two paths of length $2t+ 1$ with an edge. Can we reduce the number
of vertices in the lower bound graph by combining two paths of length $t$ each? How does this
affect our lower bound?
    \item Now, let us consider a fixed path on $t- 2$ vertices, and we add a vertex $l$ to the left end and a
vertex $r$ to the right end to make it a path on t vertices. Note that all IDs are fixed except the
IDs of $l$ and $r$. Can you show that a $t- 2$ round LOCAL algorithm will fail to 2-color such a path?
\end{enumerate}
\end{problem}

\begin{solution*}{Solution (a)}{}
Consider three paths: $P^{(1)}, P^{(2)}, P^{(3)}$ each of length $t$. Let $P^{(n)}_{i}$ denote the color of $i^{th}$ node along the $n^{th}$ path( in any arbitrary but fixed direction )
We will consider two cases:
	\begin{enumerate}[label=(\roman*)]
		\item $t = 2k + 1 $ \newline
			Let $A$ be an algorithm that can color a path in $k$ rounds. \\
			Using pigeon hole principle, we can always find two paths $P^{(i)}$ and $P^{(j)}$ such that 
			$P^{(i)}_{k+1} = P^{(j)}_{k+1}$. \\
			Construct a path $P^{(4)}$ by joining $P^{(i)}_t$ and $P^{(j)}_t$. \\
			$P^{(4)}_{k+1} = P^{(4)}_{t+k+1}  = P^{(i)}_{k+1} = P^{(j)}_{k+1}$ because $A$ is a $k$ round LOCAL algorithm. \\
			This is a contradiction since in a valid 2-coloring of $P^{(4)}$, $P^{(4)}_{k+1} \neq P^{(4)}_{t+k+1}$. \\
		\( \therefore \) The lower bound is $k+1$ to color a path of length $t$ when $t=2k+1$.
		\item $t = 2k$ \newline
			Let $A$ be an algorithm that can color a path in $k-1$ rounds. \\
			Using pigeon hole principle, we can always find two paths $P^{(i)}$ and $P^{(j)}$ such that 
			$P^{(i)}_{k+1} = P^{(j)}_{k+1}$. \\
			Construct a path $P^{(4)}$ by joining $P^{(i)}_t$ and $P^{(j)}_t$\\
			$P^{(4)}_{k+1} = P^{(4)}_{t+k+1}  = P^{(i)}_{k+1} = P^{(j)}_{k+1}$ because $A$ is a $k-1$ round LOCAL algorithm. \\
			This is a contradiction since in a valid 2-coloring of $P^{(4)}$, $P^{(4)}_{k+1} \neq P^{(4)}_{t+k+1}$. \\
		\( \therefore \) The lower bound is $k$ to color a path of length $t$ when $t=2k$. \\
		\newline
		In general, the lower bound to color a path of length $2t$ is $\frac{t+1}{2}$.
\end{enumerate}
\end{solution*}

\begin{problem}{}{problem-label-t}
\end{problem}

\begin{problem}{}{problem-label-3}
Let $T$ be a tree with $n$ nodes. The Rake and Compress
algorithm is used to heirarchically decompose $T$ into a set of paths and singleton vertices. This
is done iteratively where each iteration $i$ consists of a rake step (removing leaves) and a compress
step (removing degree-two paths) performed on all nodes of the tree. We continue until there
are no vertices left in the tree.
\begin{enumerate}[(a)]
	\item Describe how the rake and compress operations in each iteration can be implemented in
	the LOCAL model in constant rounds.
	\item Let $V_i$ be the set of nodes removed in iteration $i$, and let $T_i = V (T)\setminus (V_1\cup ...\cup V_{i-1})$ be the
	set of nodes that are not removed until the beginning of iteration i (note that $T_1 = V (T)$).
	Show that $|V_i| \geq |T_i|/2$.
	\item Prove that the total number of rounds/iterations required to reduce the tree to an empty graph is $O(log n)$.
	\item Using the output of the rake-and-compress decomposition, show that there is a deterministic algorithm in the LOCAL model to compute a 3-coloring of an unoriented tree in $O(log n)$ rounds.
\end{enumerate}
\end{problem}

\begin{solution*}{Solution (a)}{}
	\begin{algorithmic}
	\Function{delete}{u}
		\For {$v \in N(u)$}
		\State \textbf{send} delete link to $v$
		\EndFor
	\EndFunction
	\newline
	\Function{update}{u}
	\For {$v \in N(u)$}
		\If{\textbf{received} delete link from v}
			\State $N(u) \gets N(u)\setminus v$
			\State $deg(u) \gets deg(u) -1$
		\EndIf
	\EndFor
	\EndFunction
	\newline
	\State $ i \gets i^{th}$ iteration
	\State $ u \gets$ node u
	\State $ N(u) \gets$ set of adjacent nodes
	\State $ deg(u) \gets |N(u)|$

	\Comment rake step
	\If{$deg(u) = 1$}
		\State \Call{delete}{u}
	\EndIf

	\Comment compress step
	\If{$deg(u) = 2$}
		\State \Call{delete}{u}
	\EndIf
	\State \Call{update}{u}
	\end{algorithmic}
\end{solution*}

\begin{solution*}{Solution (b)}{}
	Assume the contrary.
	\[\implies |V_i| < \frac{|T_i|}{2}. \]
	Since, $T_i$ is a forest,
	\[\implies \sum_{u \in T_i}^{} deg(u) = 2(|T_i| - 1)\]
	Also,
	\[\implies \sum_{u \in T_i}^{} deg(u) \geq |V_i| + 3(|T_i| - |V_i|) \geq 2|T_i|\]
	Which is a contradiction.
	\[ \therefore |V_i| \geq \frac{|T_i|}{2}\]
\end{solution*}

\begin{solution*}{Solution (c)}{}
	Using the previous result,
	\[ |T_i| - |V_i| = |T_{i+1}| < \frac{|T_i|}{2} \]
	By recursing on $i$ we get,
	\[ |T_{i+1}| < \frac{|T_1|}{2^{i}}\]
	The algorithm ends when $|T_{n+1}| = 0$.
	\[ \implies 2^n > |T_1|\]
	\[ \therefore n = \lceil \log |T_1| \rceil \] 
	The round complexity is $O(logn)$
\end{solution*}

\begin{solution*}{Solution (d)}{}
	\begin{algorithmic}
	\State $ i \gets 1$
	\Comment iteration
	\State $ N(u) \gets$ set of nodes adjacent to $u$
	\State $ deg(u) \gets |N(u)|$
	\State $ deleted \gets false$
	\Comment whether deleted at $i^{th}$ iteration
	\State $ r \gets 0$
	\Comment round number
	\State $ S \gets \emptyset$
	\Comment adjacent nodes deleted after $u$
	\State $ E \gets \emptyset$
	\Comment adjacent nodes deleted in the same round as $u$
	\State $ color \gets ID(u) $
	\newline
	\While{$deg(u) \neq 0 \land deleted = false$}
		\Comment rake step
		\If{$deg(u) = 1$}
			\State \Call{delete}{}
		\EndIf

		\Comment compress step
		\If{$deg(u) = 2$}
			\State \Call{delete}{}
		\EndIf
		\State \Call{update}{}
	\EndWhile
	\newline
	\For{$i \in \{1...log(n)+1\}$}
		\If{$ r = i$}
			\State \Call{ColeVishkin}{}
		\EndIf
	\EndFor
	\newline
	\Function{delete}{}
		\State $deleted \gets true$
		\State $r \gets i$
		\For {$v \in N(u)$}
		\State \textbf{send} delete link to $v$
		\EndFor
	\EndFunction
	\newline
	\Function{update}{}
	\For {$v \in N(u)$}
		\If{\textbf{received} delete link from $v$}
			\State $deg(u) \gets deg(u) -1$
			\If{$r < i$}
				\State $S \gets S \cup v$
			\Else
				\State $E \gets E \cup v$
			\EndIf
		\EndIf
	\EndFor
	\EndFunction
	\newline
	\Function{ColeVishkin}{}
	\For{$ v \in S $}
		\State orient edge from $u$ to $v$
	\EndFor
	\For{$ v \in E $}
		\If{$u > v$}
			\State orient edge from $u$ to $v$
		\Else
			\State orient edge from $v$ to $u$
		\EndIf
	\EndFor \\
	Apply the Cole Vishkin algorithm on the oriented graph
	\EndFunction
	\newline
	\end{algorithmic}
A few things to note:
\begin{enumerate}
	\item $|S| + |E| \leq 2$. Can be deduced from the algorithm.
	\item Let $E_i = \{(u,v): u \in V_i, v \in V (T)\setminus (V_1\cup ...\cup V_{i-1}), (u,v) \in E(T_i)\}$ The induced subgraph $T[E_i]$ is: 
		\begin{enumerate}
			\item A forest. Since $T$ is a tree.
			\item $deg(v, T[E_i]) \leq 2 \forall v \in V(T[E_i])$. Since $|S| + |E| \leq 2$.
			\item Each connected component is thus either a singleton or a path. Cole Vishkin can be used after orienting each edge in $E_i$ according to the algorithm give above.
		\end{enumerate}
\end{enumerate}
\end{solution*}


\bibliographystyle{apalike}
\bibliography{references}

\end{document}
